% (c) 2002 Matthew Boedicker <mboedick@mboedick.org> (original author) http://mboedick.org
% (c) 2003-2007 David J. Grant <davidgrant-at-gmail.com> http://www.davidgrant.ca
% (c) 2008 Nathaniel Johnston <nathaniel@nathanieljohnston.com> http://www.nathanieljohnston.com
% (c) 2011 Scott Clark <sc932@cornell.edu> http://cam.cornell.edu/~sc932
%

%This work is licensed under the Creative Commons Attribution-Noncommercial-Share Alike 2.5 License. To view a copy of this license, visit http://creativecommons.org/licenses/by-nc-sa/2.5/ or send a letter to Creative Commons, 543 Howard Street, 5th Floor, San Francisco, California, 94105, USA.

\documentclass[letterpaper,11pt]{article}
\newlength{\outerbordwidth}
\pagestyle{empty}
\raggedbottom
\raggedright
\usepackage[UTF8]{ctex}
\usepackage[svgnames]{xcolor}
\usepackage{framed}
\usepackage{tocloft}
\usepackage{etoolbox}
\robustify\cftdotfill

%-----------------------------------------------------------
%Font
% \setCJKmainfont{Adobe Fangsong Std}
% \setCJKmainfont{YaHei Consolas Hybrid}

%-----------------------------------------------------------
%Edit these values as you see fit
\setlength{\outerbordwidth}{3pt}  % Width of border outside of title bars
\definecolor{shadecolor}{gray}{0.75}  % Outer background color of title bars (0 = black, 1 = white)
\definecolor{shadecolorB}{gray}{0.93}  % Inner background color of title bars

%-----------------------------------------------------------
%Margin setup
\setlength{\evensidemargin}{-0.25in}
\setlength{\headheight}{-0.25in}
\setlength{\headsep}{0in}
\setlength{\oddsidemargin}{-0.25in}
\setlength{\paperheight}{11in}
\setlength{\paperwidth}{8.5in}
\setlength{\tabcolsep}{0in}
\setlength{\textheight}{9.75in}
\setlength{\textwidth}{7in}
\setlength{\topmargin}{-0.3in}
\setlength{\topskip}{0in}
\setlength{\voffset}{0.1in}

%-----------------------------------------------------------
%Custom commands
\newcommand{\resitem}[1]{\item #1 \vspace{-2pt}}
\newcommand{\resheading}[1]{\vspace{8pt}
  \parbox{\textwidth}{\setlength{\FrameSep}{\outerbordwidth}
    \begin{shaded}

\setlength{\fboxsep}{0pt}\framebox[\textwidth][l]{\setlength{\fboxsep}{4pt}\fcolorbox{shadecolorB}{shadecolorB}{\textbf{\sffamily{\mbox{~}\makebox[6.762in][l]{\large #1} \vphantom{p\^{E}}}}}}
    \end{shaded}
  }\vspace{-5pt}
}

\newcommand{\ressubheading}[4]{
\begin{tabular*}{6.5in}{l@{\cftdotfill{\cftsecdotsep}\extracolsep{\fill}}r}
		\textbf{#1} & #2 \\
		\textit{#3} & \textit{#4} \\
\end{tabular*}\vspace{-6pt}}

%-----------------------------------------------------------
\begin{document}

\begin{tabular*}{7in}{l@{\extracolsep{\fill}}r}

\textbf{{\Large 乔 立}} & \textbf{\today} \\
\texttt{qiaoli.pg@gmail.com} & \\ %\texttt{blog.lipbcu.love} \\
\texttt{15026602557} & 软件工程师

\end{tabular*}

%%%%%%%%%%%%%%%%%%%%%%%%%%%%%%

\resheading{教育经历}

%%%%%%%%%%%%%%%%%%%%%%%%%%%%%%

\begin{itemize}

\item
	\ressubheading{上海交通大学}{硕士}{机械工程}{2013 - 2017}

\item
	\ressubheading{东北石油大学}{本科}{机械设计制造及其自动化}{2009 - 2013}

\end{itemize}

%%%%%%%%%%%%%%%%%%%%%%%%%%%%%%

\resheading{工作经验}

%%%%%%%%%%%%%%%%%%%%%%%%%%%%%%

\begin{itemize}

\item
	\ressubheading{安徽泰尔控股集团上海机器人有限公司}{上海}{软件工程师}{2017.7 - current}
	\begin{itemize}
      \resitem{主要完成Windows桌面软件的开发,实现机器人的运动控制程序、UI界面和相关接口的设计,同时完成相关设备的联网后台服务器开发。}
	\end{itemize}

\end{itemize}

%%%%%%%%%%%%%%%%%%%%%%%%%%%%%%

\resheading{项目经验}

%%%%%%%%%%%%%%%%%%%%%%%%%%%%%%

\begin{itemize}

\item
	\ressubheading{VR设备后台服务器开发}{}{安徽泰尔控股集团上海机器人有限公司}{2017.7 - 2018.2}
	\begin{itemize}
		\resitem{ {\bf 项目介绍:} 该项目采用Golang基于Leaf 源框架实现VR体感模拟设备的联网,实现设备认证,设备起动停止及异常运行记录收集,基于微信和支付宝支付的移动收费运行,以及RESTful后台管理接口开发。}
		\resitem{ {\bf 职责描述:} 本人的主要工作为完成系统的架构设计和功能模块划分,修改Leaf 框架实现通过TLS的加密通讯,实现设备登录和起停记录模块,并完成相应的服务器客户端的消息格式设计和 MySQL 数据库的表和字段的设计。}
	  \end{itemize}

\item
	\ressubheading{VR跳伞模拟器}{}{安徽泰尔控股集团上海机器人有限公司}{2017.1 - 2017.11}
	\begin{itemize}
		\resitem{ {\bf 项目介绍:} 该项目主要完成VR跳伞游戏的开发和游戏过程中的失重体感模拟。\\
		游戏采用 UE4 完成不同的地图的跳伞过程,实现起跳,开伞,滑翔,落地等动作。 \\
		体感模拟软件采用C++/Qt进行开发,基于EtherCAT接口实现电机的位置控制,实现游戏各个过程动作的模拟。通过串口通讯获取拉力传感器的拉力值,传递给游戏,实现角色的控制。模拟软件通过进程间通讯实现与游戏的数据交互,根据游戏状态规划出运动轨迹,实现姿态控制。}
		\resitem{ {\bf 职责描述:} 本人的主要工作为采用Qt完成体感模拟软件功能的开发与测试,以及与游戏的通讯接口的设计与实现。}
	\end{itemize}

\item
	\ressubheading{六自由度VR赛车运动模拟器}{}{安徽泰尔控股集团上海机器人有限公司}{2016.3 - 2017.1}
	\begin{itemize}
		\resitem{ {\bf 项目介绍:} 该项目通过六自由度的Stewart平台实现VR赛⻋游戏过程中的体感模拟。\\
		模拟器软件采用C++/Qt进行开发,主要功能包括通过共享内存,TCP/IP,读取目标进程内存等方式实现游戏数据的获取,体感模拟算法和运动算法的实现,基于UDP通讯的伺服驱动器控制,基于Qt的UI界面开发以及通过DLL劫持实现运动过程中的VR姿态补偿。}
		\resitem{ {\bf 职责描述:} 本人的主要工作为控制软件所有功能的开发与测试。}
	\end{itemize}

\item
	\ressubheading{空间对接半物理仿真综合试验台}{}{上海交通大学重大装备设计与控制工程研究所}{2014.7 - 2016.3}
	\begin{itemize}
		\resitem{ {\bf 项目介绍:} 该项目主要完成空间对接半物理仿真试验台的研发,完成空间⻜行器的地面对接仿真试验。主要包括九自由度的并联机器人的运动控制,动力学仿真的实现,中央控制系统人机操作界面和实时监控系统的开发。}
		\resitem{ {\bf 职责描述:} 本人的主要工作为采用LabVIEW完成上下位软件的开发,包括多系统实时通讯的实现,中央控制台人机交互界面和试验流程控制的开发,以及实时监控系统的数据采集和显示的开发。}
	\end{itemize}

\end{itemize}


%%%%%%%%%%%%%%%%%%%%%%%%%%%%%%

\resheading{技能}

%%%%%%%%%%%%%%%%%%%%%%%%%%%%%%

\begin{itemize}

\item {\bf 编程语言:} C++(熟练), Qt(熟练), Matlab(熟练), LabVIEW(精通), Go(掌握)

\item {\bf 平台:} Windows(熟练), Linux(掌握)

% \item {\bf 控制理论:} PID控制,运动轨迹规划,运动学,动力学

\end{itemize}

%%%%%%%%%%%%%%%%%%%%%%%%%%%%%%

\resheading{在校情况}

%%%%%%%%%%%%%%%%%%%%%%%%%%%%%%

\begin{itemize}

\item 2015.12 {\bf 全国研究生数学建模竞赛三等奖}

\item 2011.12 {\bf 大学英语六级}

\end{itemize}

\end{document}
